\documentclass[12pt,oneside,a4paper]{article}
\usepackage[utf8]{inputenc}
\usepackage[brazil]{babel}

\usepackage{settings/preambule}

\usepackage{listings}

\definecolor{codegreen}{rgb}{0,0.6,0}
\definecolor{codegray}{rgb}{0.5,0.5,0.5}
\definecolor{codepurple}{rgb}{0.58,0,0.82}
\definecolor{backcolour}{rgb}{0.95,0.95,0.92}

\lstdefinestyle{mystyle}{
    language=Scilab,
    backgroundcolor=\color{backcolour},   
    commentstyle=\color{codegreen},
    keywordstyle=\color{magenta},
    numberstyle=\tiny\color{codegray},
    stringstyle=\color{codepurple},
    basicstyle=\ttfamily\footnotesize,
    breakatwhitespace=false,         
    breaklines=true,                 
    captionpos=b,                    
    keepspaces=true,                 
    numbers=left,                    
    numbersep=5pt,                  
    showspaces=false,                
    showstringspaces=false,
    showtabs=false,                  
    tabsize=2,
    literate      =        % Support additional characters
      {á}{{\'a}}1  {é}{{\'e}}1  {í}{{\'i}}1 {ó}{{\'o}}1  {ú}{{\'u}}1
      {Á}{{\'A}}1  {É}{{\'E}}1  {Í}{{\'I}}1 {Ó}{{\'O}}1  {Ú}{{\'U}}1
      {à}{{\`a}}1  {è}{{\`e}}1  {ì}{{\`i}}1 {ò}{{\`o}}1  {ù}{{\`u}}1
      {À}{{\`A}}1  {È}{{\'E}}1  {Ì}{{\`I}}1 {Ò}{{\`O}}1  {Ù}{{\`U}}1
      {ä}{{\"a}}1  {ë}{{\"e}}1  {ï}{{\"i}}1 {ö}{{\"o}}1  {ü}{{\"u}}1
      {Ä}{{\"A}}1  {Ë}{{\"E}}1  {Ï}{{\"I}}1 {Ö}{{\"O}}1  {Ü}{{\"U}}1
      {â}{{\^a}}1  {ê}{{\^e}}1  {î}{{\^i}}1 {ô}{{\^o}}1  {û}{{\^u}}1
      {Â}{{\^A}}1  {Ê}{{\^E}}1  {Î}{{\^I}}1 {Ô}{{\^O}}1  {Û}{{\^U}}1
      {œ}{{\oe}}1  {Œ}{{\OE}}1  {æ}{{\ae}}1 {Æ}{{\AE}}1  {ß}{{\ss}}1
      {ç}{{\c c}}1 {Ç}{{\c C}}1 {ø}{{\o}}1  {å}{{\r a}}1 {Å}{{\r A}}1
      {ã}{{\~a}}1  {õ}{{\~o}}1  {Ã}{{\~A}}1 {Õ}{{\~O}}1
      {ñ}{{\~n}}1  {Ñ}{{\~N}}1  {¿}{{?`}}1  {¡}{{!`}}1
      {°}{{\textdegree}}1 {º}{{\textordmasculine}}1 {ª}{{\textordfeminine}}1
      % ¿ and ¡ are not correctly displayed if inconsolata font is used
      % together with the lstlisting environment. Consider typing code in
      % external files and using \lstinputlisting to display them instead. 
}

\lstset{style=mystyle}

\usepackage{listings}

%\definecolor{codegreen}{rgb}{0,0.6,0}
\definecolor{codegreen}{RGB}{100,174,100}
\definecolor{codegray}{rgb}{0.5,0.5,0.5}
\definecolor{codepurple}{rgb}{0.58,0,0.82}
\definecolor{backcolour}{rgb}{0.95,0.95,0.92}
%\definecolor{backcolour}{RGB}{247,247,242}
%\definecolor{backcolour}{RGB}{252, 252, 247}

\definecolor{codeblue}{RGB}{50,185,185}
\definecolor{stringcolor}{RGB}{188,143,143}
\definecolor{emphcolor}{RGB}{176,24,19}
\definecolor{conditional}{RGB}{160,32,240}
\definecolor{variables}{RGB}{131,67,16}
\definecolor{brackets}{RGB}{74,85,219}
\definecolor{stops}{RGB}{95,158,160}
\definecolor{infinity}{RGB}{218,112,214}
\definecolor{operators}{RGB}{92,92,92}
\definecolor{colon}{RGB}{255,170,0}

\renewcommand{\ttdefault}{pcr}

\lstdefinestyle{scilab_colors}{
    language=Scilab,
    %frame=single,
    backgroundcolor=\color{backcolour},   
    commentstyle=\color{codegreen},
    %keywordstyle=\color{magenta},
    keywordstyle=\color{codeblue},
    numberstyle=\tiny\color{codegray},
    %identifierstyle=\color{blue},
    %stringstyle=\color{codepurple},
    stringstyle=\color{stringcolor},
    basicstyle=\ttfamily\footnotesize,
    emph={function,endfunction},
    emphstyle=\color{emphcolor},
    emph={[2]for, while, if, then, end},
    emphstyle={[2]\color{conditional}},
    emph={[3]lambda, x1, k, n_erro, A, x0, epsilon, alpha, M},
    emphstyle={[3]\bfseries\color{variables}}, 
    emph={[4]\$},
    emphstyle={[4]\color{colon}},
    emph={[5]Metodo_Potencia, Metodo_Potencia_2, Potencia_deslocada_inversa, Potencia_deslocada_Rayleigh},
    emphstyle={[5]\underbar},
    emph={[6]break, continue, return},
    emphstyle={[6]\color{stops}},
    emph={[7]\%inf},
    emphstyle={[7]\color{infinity}},
    emph={[8]Resolve_com_LU},
    emphstyle={[8]\bfseries\color{codepurple}},
    breakatwhitespace=false,         
    breaklines=true,                 
    captionpos=b,                    
    keepspaces=true,                 
    numbers=left,                    
    numbersep=5pt,                  
    showspaces=false,                
    showstringspaces=false,
    showtabs=false,                  
    tabsize=2,
    literate      =        % Support additional characters
      {á}{{\'a}}1  {é}{{\'e}}1  {í}{{\'i}}1 {ó}{{\'o}}1  {ú}{{\'u}}1
      {Á}{{\'A}}1  {É}{{\'E}}1  {Í}{{\'I}}1 {Ó}{{\'O}}1  {Ú}{{\'U}}1
      {à}{{\`a}}1  {è}{{\`e}}1  {ì}{{\`i}}1 {ò}{{\`o}}1  {ù}{{\`u}}1
      {À}{{\`A}}1  {È}{{\'E}}1  {Ì}{{\`I}}1 {Ò}{{\`O}}1  {Ù}{{\`U}}1
      {ä}{{\"a}}1  {ë}{{\"e}}1  {ï}{{\"i}}1 {ö}{{\"o}}1  {ü}{{\"u}}1
      {Ä}{{\"A}}1  {Ë}{{\"E}}1  {Ï}{{\"I}}1 {Ö}{{\"O}}1  {Ü}{{\"U}}1
      {â}{{\^a}}1  {ê}{{\^e}}1  {î}{{\^i}}1 {ô}{{\^o}}1  {û}{{\^u}}1
      {Â}{{\^A}}1  {Ê}{{\^E}}1  {Î}{{\^I}}1 {Ô}{{\^O}}1  {Û}{{\^U}}1
      {œ}{{\oe}}1  {Œ}{{\OE}}1  {æ}{{\ae}}1 {Æ}{{\AE}}1  {ß}{{\ss}}1
      {ç}{{\c c}}1 {Ç}{{\c C}}1 {ø}{{\o}}1  {å}{{\r a}}1 {Å}{{\r A}}1
      {ã}{{\~a}}1  {õ}{{\~o}}1  {Ã}{{\~A}}1 {Õ}{{\~O}}1
      {ñ}{{\~n}}1  {Ñ}{{\~N}}1  {¿}{{?`}}1  {¡}{{!`}}1 {[}{{\textcolor{brackets}{[}}}{1}
        {]}{{\textcolor{brackets}{]}}}{1}
        {(}{{\textcolor{brackets}{(}}}{1}
        {)}{{\textcolor{brackets}{)}}}{1} 
        {+}{{{\color{operators}+}}}1
        {-}{{{\color{operators}-}}}1
        {*}{{{\color{operators}*}}}1
        {/\ }{{{\color{operators}/}}}1
        {=}{{{\color{operators}=}}}1
        {^}{{{\color{operators}\textasciicircum}}}1
        {<}{{{\color{operators}<}}}1
        {>}{{{\color{operators}>}}}1
        %{&}{{{\color{operators}\&}}}1
        %{!=}{{{\color{codegreen}$\neq$}}}1
        %{~}{{{\color{operators}\textasciitilde}}}{1}
        %{:}{{{\color{colon}:}}}{1}
      {°}{{\textdegree}}1 {º}{{\textordmasculine}}1 {ª}{{\textordfeminine}}1
      % ¿ and ¡ are not correctly displayed if inconsolata font is used
      % together with the lstlisting environment. Consider typing code in
      % external files and using \lstinputlisting to display them instead.
}

%\lstset{style=scilab_colors}

\usepackage{etoolbox} % provides the \patchcmd macro
\makeatletter
\patchcmd{\lsthk@SelectCharTable}{`)}{``}{}{} % patch listings
%\patchcmd{\lsthk@SelectCharTable}{``}{`)}{}{} % undo patch if needed
\makeatother
\usepackage{hhline}

\newcommand{\fontcode}[2]{{\fontfamily{#1}\selectfont #2}}

%\title{Relatório - AP1_ALNum}
\title{Relatório\\Computação Escalável}
\student{Tiago da Silva \\João Alcindo Ribeiro de Azevedo\\Germano Andrade Brandão}
\local{Rio de Janeiro}

\coversheet{}%write 'yes' or left it blank

\professor{.}
\summary{.}

\begin{document}

\maketitle

\tableofcontents
\newpage


\section{Introdução}
    Trataremos aqui das decisões de projetos adotadas ao longo do trabalho, procurando esclarecer o nosso ponto de vista acerca de cada escolha. Além disso, toda a estrutura será detalhada, de forma a ser possível entender bem como funciona todo o processo de simulação de ponta a ponta. Por fim, mostraremos como o trabalho foi adaptado para que fosse processado em nuvem, utilizando majoritariamente os serviços da Amazon AWS.

\section{Pipeline}

	\begin{figure} 
		\centering 
		\includegraphics[width=\textwidth]{../images/relational_diagram.png} 
		\caption{Modelo relacional do banco de dados utilizado para a captura persistente dos dados gerados pela simulação.} 
		\label{fig:relational} 
	\end{figure} 

	Escolhemos uma pipeline de extração, transformação e carregamento (ETL, em oposição a ELT); importantemente, esta escolha está amarrada à contemplação de que os procedimentos para serializar os dados e os tornar receptivos a um banco de dados relacional gozam de intensidade computacinoal enxuta. Utilizamos, para isso, um banco de dados alicerçado no modelo relacional da Figura~\ref{fig:relational}. Nas seções seguintes, desta maneira, descrevemos a implementação de cada estágio de nosso pipeline ETL. 
	
	Neste ínterim, advogamos o modelo escolhido e apresentamos o framework geral para controlar a comunicação entre os processos responsáveis pelo inserção dos dados em um banco de dados. Inicialmente, as tabelas \texttt{scenarios}, \texttt{ants}, \texttt{anthills} e \texttt{foods} almejam incorporar as informações brutas geradas pelas simulações; isso enseja a expansão da amplitude de análises subsequentes. As tabelas com prefixo \texttt{stats\_}, por outro lado, são atualizadas periodicamente e permitem, desta maneira, o acesso, em diferentes aspectos de granularidades\footnote{Explicitamente, escrevemos \texttt{global} para as análises globais, que tangenciam todos os cenários executados; \texttt{local}, para as análises que consolidam os aspectos de cada cenário; e \texttt{atomic}, para as análises que conformam as instâncias que apontam para os formigueiros.}, às informações por usuários (na Seção~\ref{sec:user}, exploramos o desenho de uma interface que direciona o usuário ao exame dos cenários correntes e passados). Alternativamente, poderíamos 

	\begin{enumerate} 
		\item implementar uma base de dados analítica, com um modelo dimensional que combinasse parciomoniosamente as tabelas para o cômputo das quantidades solicitadas nos requisitos da modelagem;
		\item e, mais disruptivamente, optar pela utilização de um banco de dados noSQL, com um pipeline ELT, em que os dados seriam armazenados e, em seguida, processados. 
	\end{enumerate} 

	\noindent Nestas condições, nossas escolhas estão amarradas a compleições objetivas: com respeito a 1, precisaríamos comportar, em nosso sistema, um mecanismo de transferência de informações entre um banco de dados operacional e um analítico, trascendendo, em nossa opinião, o objetivo desta avaliação, que não inclui a modelagem dimensional; com respeito a 2, nossas breves exposições a bancos de dados noSQL (como o MongoDB) culminaram em nossas reticências em os introduzir neste sistema. Neste cenário, nossas escolhas por um banco de dados relacional estão alicerçadas em sua conveniência e na existência de múltiplas ferramentas que permitem a sua integração em um sistema distribuído. 

	Adicionalmente, a distribuição da pipeline ETL em múltiplos nós de computação implicou a escolha de um mecanismo de comunicação; em nossa opinião, este mecanimos deveria (1) ser escalável (isto é, as mensagens, se existirem, têm de ser persistentes e as chamadas, assíncronas; assim, garantimos que os clientes podem interagir com a interface e que seus dados serão inseridos na base de dados), (2) ser transparente (equivalentemente, a sua incorporação ao sistema existente deveria exigir modificações enxutas no código-fonte) e (3), mais importante, compatível com um cenário em que a estrutura subjacente das mensagens (como o tipo de dado e o tamanho) é difusa e variável. Vislumbramos, nestas condições, um tripleto de procedimentos, que descrevemos em seguida.

	\begin{enumerate} 
		\item \textit{Remote Procedure Calls}. Apesar de as chamadas remotas permitirem transparência na comunicação e serem compatíveis com o nosso cenário, elas não são apropriadas em um cenário em que não podemos garantir que o receptor será executado quando o requerimento for enviado, como o nosso, em que múltiplos clientes podem interagir com uma quantidade enxuta de servidores. 
		\item \textit{Message Passing Interface}. A interface de passagem de mensagens, por outro lado, não é compatível com o sistema que desejamos, na medida em que sua utilização está amarrada à estimativas do tamanho da mensagem, que, em nosso cenário, não estão disponíveis. Mais crucialmente, a introdução desse mecanismo exige, em alguns casos, a escolha cautelosa dos processos responsáveis pelo processamento das mensagens. Logo, em oposição à sua abstração conveniente e à sua flexibilidade, esta interface é inadequada para o que almejamos. 
		\item \textit{Publish-Subscribe}. Nestas circunstâncias, o publish-subscribe permite, em oposição às chamadas de procedimento remotas, o processamento subsequente de mensagens, que são persistentes, em um momento posterior ao seu envio pelo produtor (comunicação assíncrona); em oposição à interface de passagem de mensagens, ele é consistente com o nosso sistema, com uma implementação que não subverte o código-fonte inicial. 
	\end{enumerate} 

	\noindent Neste sentido, escolhemos o mecanismo de comunicação por mensagens persistentes publish-subscribe para o gerenciamento da comunicação entre os nós de processmaneto. Utilizamos, para isso, a biblioteca \texttt{Celery}, em \texttt{Python}, que providencia uma implementação eficiente deste mecanismo. Precisávamos, assim, de um broker para garantir a integração consistente de múltiplas máquninas em um sistema distribuído; escolhemos, portanto, a implementação do \texttt{RabbitMQ} por sua popularidade\footnote{Serviços populares, com uma comunidade mais extensa, permitem que usuários possam instanciar uma rede de apoio mútuo que incrementa a eficiência do desenvolvimento de \textit{software}.} e por sua acessibilidade em servições de computação em nuvem, como a AWS (AmazonMQ; veja a Seção~\ref{sec:mq}). 

	
	Nas seções seguintes, desta maneira, descreveremos os procedimentos utilizados para extração dos dados da simulação, sua subsequente transformação em um formato compatível com um banco de dados relacional, e seu carregamento em tabelas. Conforme enfatizamos na Seção~\ref{sec:rds}, utilizamos o serviço RDS da AWS para a implementação do banco de dados, que é escalável e contempla as operações do PostgreSQL, com uma interface adequada em \texttt{Python}. 

\subsection{Extração}

Logo, com o objetivo de descrever a extração dos dados gerados pelos clientes, encetamos com uma descrição de seu funcionamento. Explicitamente, implementamos uma simulação de um sistema emergente, em que formigas de distintos formigueiros interagem com um mapa finito para capturar alimentos e os conduzir a seus depósitos. Cada simulação, neste caso, é caracterizada por um conjunto de informações, como a quantidade de formigas e a extensão da execução, que precisamos extrair e direcionar a um consumidor que as alocará em uma base de dados. A etapa de extração, deste modo, consistiu na aplicação de procedimentos que garantissem a consistência entre as implementações dos mecanismos de comunicação utilizados, o \texttt{Celery}, e os processos de geração de dados de nossa implementação. Especificamente, serializamos os atributos das classes subjacentes à simulação e utilizamos o mecanismo de comunicação para transferir os dados do JSON culminante para o servidor, em que, conforme descrevemos na seção seguinte, eles são transformados em um formato cartesiano. 

\subsection{Transformação}

Nestas veredas, a etapa de transformação está amarrada à captura das mensagens empilhadas no broker e seu subsequente processamento, com a modificação de um JSON não estruturado para enformar instâncias compatíveis com o modelo relacional da Figura~\ref{fig:relational}. Dicotomicamente, este procedimento é computacionalmente enxuto, na medida em que ele é equivalente à remodelação do formato de uma JSON, e, em contraste, ele é volumoso: em cada iteração da simulação, delineada como um momento em que todas as formigas executaram suas ações, os clientes se comunicam com o broker, e o intervalo entre mensagens sequentes está na ordem de milissegundos. Importantemente, esta verificação estende nossas motivações para a escolha de um mecanismo de comunicação assíncrono e orientado a mensagens persistentes, como o publish-subscribe. 

\subsection{Carregamento}


\section{Armazenamento em nuvem (AWS)}
    O processamento e o armazenamento locais podem ser vantajosos por ter uma velocidade alta e não depender de conexões externas. No entanto, essas vantagens podem acabar rapidamente ao passo que a quantidade a ser processada/armazenada começa a aumentar. Nesse sentido, uma das alternativas mais adotadas atualmente é a de computação em nuvem, onde um ambiente vasto com soluções prontas ou customizáveis são disponibilizadas e de forma escalável, de acordo com a demanda.
    
    Dito isso, para esse trabalho utilizamos alguns dos serviços de nuvem da Amazon Web Services, a fim de agregar mais robustez ao nosso pipeline de dados. Assim, veremos agora quais recursos foram utilizados e informações das instâncias criadas em cada um deles.

\subsection{AmazonMQ} \label{sec:mq} 
    Uma vez que utilizamos o RabbitMQ como broker, temos o AmazonMQ como serviço de gerenciamento de mensagens na AWS para criar uma instância com o RabbitMQ.
    
    Ao instanciar com as opções devidamente escolhidas, ficamos com as seguintes informações
    \begin{table}[!ht]
        \centering
        \begin{tabular}{|c|c|c|}\hline
            Broker engine & Engine version & Instance type\\
            \hline
            \fontcode{lmtt}{RabbitMQ} & \fontcode{lmtt}{3.9.16} & \fontcode{lmtt}{mq.t3.micro}\\
            \hline
        \end{tabular}
        \caption{Informações da instância na AmazonMQ}
        \label{tab:amazonmq}
    \end{table}
    
    A partir disso, podemos nos conectar com o broker a partir dos dados a seguir.
    \begin{table}[!ht]
        \centering
        \begin{tabular}{|c|c|c|}\hline
            ARN (Amazon Resource Name) &  RabbitMQ web console & port\\\hline
            \begin{minipage}{.5\textwidth}\vspace{1mm}\fontcode{lmtt}{arn:aws:mq:us-east-1:676432491375\\:broker:TJG:b-7182fca9-4c07-4bfa-b\\e01-72310cb18d60}\vspace{1mm}\end{minipage} & \href{https://b-7182fca9-4c07-4bfa-be01-72310cb18d60.mq.us-east-1.amazonaws.com}{\begin{minipage}{.55\textwidth}\fontcode{lmtt}{https://b-7182fca9-4c07-4bfa-be01-723\\10cb18d60.mq.us-east-1.amazonaws.com}
            \end{minipage}} & \fontcode{lmtt}{5671}\\\hline
        \end{tabular}
        \caption{Informações para conexão}
        \label{tab:amazonMQconn}
    \end{table}
    
    Para acessar o console, basta utilizar as credenciais a seguir.
    \begin{table}[!ht]
        \centering
        \begin{tabular}{|c|c|}\hline
            Username & Password \\\hline
            \fontcode{lmtt}{username} & \fontcode{lmtt}{passwordpassword} \\\hline
        \end{tabular}
        \caption{Credenciais RabbitMQ}
        \label{tab:rabbitMQcredentials}
    \end{table}
   
    
    

\subsection{RDS} \label{sec:rds} 
    
    
\subsection{ECR}
    Para hospedar imagens Docker, utilizamos o Amazon ECR (\textit{Elastic Container Registry}) que é um registro de contêiner totalmente gerenciado da Amazon AWS.
    
    Nesse sentido, criamos dois repositórios, um para o \textit{celery} e outro para o \textit{spark}.
    
    \begin{table}[!ht]
        \centering
        \begin{tabular}{|c|c|}\hline
            Repository name &  URI\\\hhline{|=|=|}
            \fontcode{lmtt}{tjg-celery} &  \fontcode{lmtt}{676432491375.dkr.ecr.us-east-1.amazonaws.com/tjg-celery}\\\hline
            \fontcode{lmtt}{tjg-spark} & \fontcode{lmtt}{676432491375.dkr.ecr.us-east-1.amazonaws.com/tjg-spark}\\\hline
        \end{tabular}
        \caption{Informações das imagens}
        \label{tab:amazonECR}
    \end{table}
\subsection{ECS}
    Para gerenciar os containers, utilizamos o Amazon ECS (\textit{Elastic Container Service}), onde podemos executar tarefas dentro do cluster.
    
    \begin{table}[!ht]
        \centering
        \begin{tabular}{|c|c|c|}\hline
            Cluster & Cluster ARN & Launch type\\\hline
            \fontcode{lmtt}{TJG} & \fontcode{lmtt}{arn:aws:ecs:us-east-1:676432491375:cluster/TJG} & \fontcode{lmtt}{FARGATE}\footnotemark \\\hline
        \end{tabular}
        \caption{Informações do cluster}
        \label{tab:ino_amazonECS}
    \end{table}
    \footnotetext{O \href{https://docs.aws.amazon.com/pt_br/AmazonECS/latest/developerguide/AWS_Fargate.html}{AWS Fargate} é uma tecnologia que pode ser usada com o Amazon ECS para executar contêineres sem a necessidade de gerenciar servidores ou clusters de instâncias do Amazon EC2.}
    
    Com isso, ficamos aptos a definir as \textit{tasks}
     
\section{Interface com o usuário} \label{sec:user} 

\end{document}

